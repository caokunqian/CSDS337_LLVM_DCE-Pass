%%
%% This is file `sample-sigplan.tex',
%% generated with the docstrip utility.
%%
%% The original source files were:
%%
%% samples.dtx  (with options: `sigplan')
%% 
%% IMPORTANT NOTICE:
%% 
%% For the copyright see the source file.
%% 
%% Any modified versions of this file must be renamed
%% with new filenames distinct from sample-sigplan.tex.
%% 
%% For distribution of the original source see the terms
%% for copying and modification in the file samples.dtx.
%% 
%% This generated file may be distributed as long as the
%% original source files, as listed above, are part of the
%% same distribution. (The sources need not necessarily be
%% in the same archive or directory.)
%%
%% Commands for TeXCount
%TC:macro \cite [option:text,text]
%TC:macro \citep [option:text,text]
%TC:macro \citet [option:text,text]
%TC:envir table 0 1
%TC:envir table* 0 1
%TC:envir tabular [ignore] word
%TC:envir displaymath 0 word
%TC:envir math 0 word
%TC:envir comment 0 0
%%
%%
%% The first command in your LaTeX source must be the \documentclass command.
\documentclass[sigplan,screen]{acmart}
%% NOTE that a single column version is required for 
%% submission and peer review. This can be done by changing
%% the \doucmentclass[...]{acmart} in this template to 
%% \documentclass[manuscript,screen,review]{acmart}
%% 
%% To ensure 100% compatibility, please check the white list of
%% approved LaTeX packages to be used with the Master Article Template at
%% https://www.acm.org/publications/taps/whitelist-of-latex-packages 
%% before creating your document. The white list page provides 
%% information on how to submit additional LaTeX packages for 
%% review and adoption.
%% Fonts used in the template cannot be substituted; margin 
%% adjustments are not allowed.
%%
%% \BibTeX command to typeset BibTeX logo in the docs
\AtBeginDocument{%
  \providecommand\BibTeX{{%
    \normalfont B\kern-0.5em{\scshape i\kern-0.25em b}\kern-0.8em\TeX}}}

%% Rights management information.  This information is sent to you
%% when you complete the rights form.  These commands have SAMPLE
%% values in them; it is your responsibility as an author to replace
%% the commands and values with those provided to you when you
%% complete the rights form.
%\setcopyright{acmcopyright}
\copyrightyear{2022}
\acmYear{2022}

%% These commands are for a PROCEEDINGS abstract or paper.
\acmConference[Compiler Design]{May 3, 2024}{Cleveland, OH}
%
%  Uncomment \acmBooktitle if th title of the proceedings is different
%  from ``Proceedings of ...''!
%

%%
%% Submission ID.
%% Use this when submitting an article to a sponsored event. You'll
%% receive a unique submission ID from the organizers
%% of the event, and this ID should be used as the parameter to this command.
%%\acmSubmissionID{123-A56-BU3}

%%
%% For managing citations, it is recommended to use bibliography
%% files in BibTeX format.
%%
%% You can then either use BibTeX with the ACM-Reference-Format style,
%% or BibLaTeX with the acmnumeric or acmauthoryear sytles, that include
%% support for advanced citation of software artefact from the
%% biblatex-software package, also separately available on CTAN.
%%
%% Look at the sample-*-biblatex.tex files for templates showcasing
%% the biblatex styles.
%%

%%
%% The majority of ACM publications use numbered citations and
%% references.  The command \citestyle{authoryear} switches to the
%% "author year" style.
%%
%% If you are preparing content for an event
%% sponsored by ACM SIGGRAPH, you must use the "author year" style of
%% citations and references.
%% Uncommenting
%% the next command will enable that style.
%%\citestyle{acmauthoryear}

%%
%% end of the preamble, start of the body of the document source.
\begin{document}

%%
%% The "title" command has an optional parameter,
%% allowing the author to define a "short title" to be used in page headers.
\title{Dead Code Elimination} 
%% Of note is the shared affiliation of the first two authors, and the
%% "authornote" and "authornotemark" commands
%% used to denote shared contribution to the research.
\author{Carter Chen}
\email{cjc230@case.edu}
\affiliation{%
  \institution{Case Western Reserve University}
  \city{Cleveland}
  \state{Ohio}
  \country{USA}
}

\author{Jacob Bair}
\email{jeb287@case.edu}
\affiliation{%
  \institution{Case Western Reserve University}
  \city{Cleveland}
  \state{Ohio}
  \country{USA}
}

\author{Jack Qian}
\email{cxq@case.edu}
\affiliation{%
  \institution{Case Western Reserve University}
  \city{Cleveland}
  \state{Ohio}
  \country{USA}
}

%%
%% The abstract is a short summary of the work to be presented in the
%% article.
\begin{abstract}
  This project takes current LLVM code and removes "redundant" code in an attempt to make LLVM more efficient. Implementation <explanation>
\end{abstract}


%%
%% Keywords. The author(s) should pick words that accurately describe
%% the work being presented. Separate the keywords with commas.
\keywords{LLVM, optimization pass, dead code elimination, compiler design}


%%
%% This command processes the author and affiliation and title
%% information and builds the first part of the formatted document.
\maketitle

\section{Introduction}
    Dead Code Elimination (DCE) is an optimization pass for LLVM, in which code that is either redundant or unachievable through most, if not any means is removed. Dead code can exist in programming in general in the event that the code that is intended to run fails, or if the person originally creating the compiler forgot to remove it. While usually a good idea to have a backup in the event the main code does not work, over the course of continuous usage, dead code can lead to a lot of time wasted, if not removed beforehand.\\
    \intent While it might seem that dead code will take up a small amount of computing power, in the event that the area of code is looped through several times, the impact of a small time loss can be compounded, sometimes exponentially. Saving time when possible should take priority before thinking of expanding the hardware requirements for a device, when running a compiler or any program.
    
    \subsection{Unused Statements}
    \begin{verbatim}
        Func add(x, y) {
        x + y = z; // Unused statement
        return x + y;
	    }
    \end{verbatim}
    In this example, "x + y = z;" is used presumably to be returned by the "add" function, however "x + y" is returned instead. The value "z" is never used, and the line can be removed.
    
    \subsection{Unreachable Statements}
    \begin{verbatim}
        Func add(x, y) {
        return x + y;
        x + y = z; // Unreachable statement
        return z; // Unreachable statement
        }
    \end{verbatim}
    In this example, "x + y = z;" and "return z;" are unreachable, since "return x + y;" ends the function's operations by returning the sum of x + y.
    
    \subsection{Constant Folding}
    \begin{verbatim}
        Func addWithZ(x, y) {
        z = 5 + 3 * 0; // Redundant code here
        return x + y + z;
        }
    \end{verbatim}
    Since "z" will compute to 5 (since 3 * 0 = 0, and 5 + 0 = 5), the "+ 3 * 0;" persists as redundant code, asking the compiler to compute unnecessarily.

    \subsection{Unused Function}
    \begin{verbatim}
        Func add(x, y) { // This function is unused
        return x + y;
        }
    \end{verbatim}
    If a function is never called but exists within the codebase, it remains redundant, and should be removed.

\section{Project Scope}
    There are several more kinds of DCE that were not shown in the introduction prior, so for the scope of our project, we decided to narrow down the types of DCE we would implement. There are also most likely a gratuitous amount of areas in which we could implement DCE, however due to the time restraints for our project, we have no intention of implementing every possible DCE within LLVM.

\section{Implementation}
    Implementation w/ examples

\section{Testing}
    Testing w/ examples
    
\section{Result}


%%
%% The acknowledgments section is defined using the "acks" environment
%% (and NOT an unnumbered section). This ensures the proper
%% identification of the section in the article metadata, and the
%% consistent spelling of the heading.
\begin{acks}
    To Aaron for the tutorial github repository and the very detailed AST explanation
\end{acks}

%%
%% The next two lines define the bibliography style to be used, and
%% the bibliography file.
\bibliographystyle{ACM-Reference-Format}
\bibliography{sample-base}
\end{document}
\endinput
%%
%% End of file `sample-sigplan.tex'.